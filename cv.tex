\documentclass[a4paper,10pt]{article}

%A Few Useful Packages
\usepackage{marvosym}
\usepackage{fontspec} 					%for loading fonts
\usepackage{xunicode,xltxtra,url,parskip} 	%other packages for formatting
\RequirePackage{color,graphicx}
\usepackage[usenames,dvipsnames]{xcolor}
\usepackage[big]{layaureo} 				%better formatting of the A4 page
% an alternative to Layaureo can be ** \usepackage{fullpage} **
\usepackage{supertabular} 				%for Grades
\usepackage{titlesec}					%custom \section
\usepackage{longtable}
\usepackage{booktabs}

%Setup hyperref package, and colours for links
\usepackage{hyperref}
\definecolor{linkcolour}{rgb}{0,0.2,0.6}
\hypersetup{colorlinks,breaklinks,urlcolor=linkcolour, linkcolor=linkcolour}

%FONTS
\defaultfontfeatures{Mapping=tex-text}
\setmainfont[SmallCapsFont = Fontin-SmallCaps.ttf, ItalicFont = Fontin-Italic.ttf, BoldFont = Fontin-Bold.ttf]{Fontin-Regular.ttf}

%CV Sections inspired by:
%http://stefano.italians.nl/archives/26
\titleformat{\section}{\Large\scshape\raggedright}{}{0em}{}[\titlerule]
\titlespacing{\section}{0pt}{3pt}{3pt}
%Tweak a bit the top margin
%\addtolength{\voffset}{-1.3cm}

%Italian hyphenation for the word: ''corporations''
%\hyphenation{im-pre-se}

%-------------WATERMARK TEST [**not part of a CV**]---------------
%\usepackage[absolute]{textpos}
%
%\setlength{\TPHorizModule}{30mm}
%\setlength{\TPVertModule}{\TPHorizModule}
%\textblockorigin{2mm}{0.65\paperheight}
%\setlength{\parindent}{0pt}

%--------------------BEGIN DOCUMENT----------------------
\begin{document}

\pagestyle{empty} % non-numbered pages

\font\fb=''[cmr10]'' %for use with \LaTeX command

%--------------------TITLE-------------
\par{\centering
		{\Huge Ufuk \textsc{Kayserilioglu}
	}\bigskip\par}

%--------------------SECTIONS-----------------------------------
%Section: Personal Data
\section{Personal Data}

\begin{tabular}{rl}
    \textsc{Place and Date of Birth:} & Istanbul, Turkey  - 27 July 1975 \\
    \textsc{Address:}                 & Emanetci Sok. No: 1/7 Cihangir, Istanbul, TURKEY \\
    \textsc{Phone:}                   & +90 532 6644030\\
    \textsc{email:}                   & \href{mailto:ufuk@paralaus.com}{ufuk@paralaus.com}
\end{tabular}

%Section: Work Experience at the top
\section{Work Experience}
\begin{tabular}{r|p{11cm}}
  \emph{Current}                & Managing Partner and CTO at \href{http://www.enkuba.com}{\textsc{Enkuba}}, Istanbul \\
  \textsc{Aug 2011 - Present}   & \emph{Startup Accelerator / Incubator} \\

  \multicolumn{2}{c}{} \\

  \textsc{Jan 2002 - Aug 2011}  & Senior Software Engineer at \href{http://www.phonoclick.com}{\textsc{PhonoClick}}, Istanbul \\
                                & \emph{Customer Interaction Software}  \\

  \multicolumn{2}{c}{} \\
 
  \textsc{Aug 2001 - Dec 2001}  & Software Engineer at \href{http://www.parkyeri.com}{\textsc{Parkyeri Yazılım}}, Istanbul \\ 
                                & \emph{Software Development, Operations and Consultancy.} \\

  \multicolumn{2}{c}{} \\

  \textsc{Dec 2000 - Apr 2001}  & Consultant Software Engineer at \href{http://turk.net/}{\textsc{Turk.Net}}, Istanbul \\ 
                                & \emph{Internet Service Provider} \\

  \multicolumn{2}{c}{} \\

  \textsc{Sep 1997 - Jun 2004}  & Teaching Assistant at \href{http://www.boun.edu.tr/}{\textsc{Bogazici University}}, Istanbul \\ 
                                & \emph{Physics Department} \\

  \multicolumn{2}{c}{} \\

\end{tabular}

%Section: Education
\section{Education}
\begin{longtable}{r|p{11cm}} 
  \textsc{August} 2005  & MSc and PhD in \textsc{Physics} \\
                        & \textbf{Bogazici University}, Istanbul\\
                        & Thesis: ``Quantum Group Structures Associated with Invariances of Some Physical Algebras'' \\
                        & \small Advisor: Prof. Metin \textsc{Arik}\\
  \multicolumn{2}{c}{} \\

  \textsc{July} 1998    & Bachelor of Science, \textsc{Physics} \\
                        & Bachelor of Science, \textsc{Mathematics} (double major) \\
                        & \normalsize\textbf{Bogazici University}, Istanbul \\
                        & \normalsize \textsc{Gpa}: 3.63/4.0 \\
  \multicolumn{2}{c}{} \\

  \textsc{July} 1993    & High School \\
                        & \normalsize\textbf{Robert College}, Istanbul \\
  \multicolumn{2}{c}{} \\
\end{longtable}

\newpage 
%Section: Work Details
\section{Work Details}
\begin{longtable}{r|p{11cm}}
  \multicolumn{2}{l}{\textbf{Enkuba \footnotesize{Aug 2011 - Present}}} \\
  \specialrule{.01em}{0.5em}{1em}

  \textsc{Aug 2011 - Mar 2013}  & Cofounded the company and assumed role as Managing Partner. In that role, my responsibilities were to get in touch with the local startup ecosystem, recruit promising startups into the accelerator programme, screen potential programme applications, make feasibility studies on candidate startups and make investment decisions. In addition to these management roles, I was also mentoring founders in the programme on topics ranging from lean startup methodologies to agile development and helping them with their investor pitches. \\
                                & \\
                                & As CTO of the incubator, I helped founders in the programme tackle programming challenges with hands-on coding sessions. For one startup (\href{http://www.bukacparaeder.com}{BuKacParaEder}) with no technical co-founder, I built the initial website from the ground up using Ruby on Rails. That particular website, an online antiques valuation site, was a fully operational internet services project complete with payment systems integration (PayPal and Virtual POS), mini CMS, PDF report generation and backend administrative management interface built in two languages (Turkish and English). \\
  \multicolumn{2}{c}{} \\

  \textsc{Mar 2013 - Present}   & After 6 startup investments, partners decided to pivot the business to incubating its own projects in house. The first (and current) project was to build an online insurance broker, \href{https://sigortaci123.com}{Sigortaci123}. Our observation was that competition in the local market were offering complex and confusing solutions and, most of the time, they were steering customers towards finalizing the transaction via a call-center operator. Our goal was to utilize technology as much as possible to discover information about the customer rather than asking and to provide a pleasant easy-to-use user interface that would steer them towards an online purchase. \\
                                & \\
                                & As CTO, I was responsible for getting in touch with the underwriting insurance companies to negotiate access to their web APIs, building a Ruby gem abstracting the APIs of different companies into a single unified interface and constructing a Ruby on Rails, Bootstrap, Angular website - both backend and frontend. It was decided that the project will have a different exact-match domain name for each insurance type, so the Motor Own Damage insurance is being hosted at \href{http://kasko123.com}{one domain} and the Third Party Compulsory Liability motor insurance is being hosted at \href{http://trafik123.com}{at another}, all being operated by the same Rails application under the hood. \\
                                & \\
                                & The project has been open to public since the beginning of December with 4 insurance company integrations and is in public beta status. Since launch, we have recruited one junior Rails developer and we have been running A/B tests, analysing the analytics reports, optimising the UI flow, implementing features to increase conversion, talking to customers online to better understand their needs and building integrations with more insurance companies. \\
  \multicolumn{2}{c}{} \\

  \newpage

  \multicolumn{2}{l}{\textbf{PhonoClick \footnotesize{Jan 2002 – Aug 2011}}} \\
  \specialrule{.01em}{0.5em}{1em}

  \textsc{2002}                 & Joined the company as a member of the founding team. \\
  \multicolumn{2}{c}{} \\

  \textsc{2002-2011}            & Initiated, and later on led, the development of PhonoClick Telephony Server; a C/C++ based, VoiceXML 2.0 compliant voice interaction system that works over POTS, ISDN and VoIP lines and is capable of utilizing ASR
                                and TTS resources in various languages. As the lead developer of the product, I was responsible for the architecture, design and implementation of all aspects of the system. Constantly communicated with the sales team about business requests and worked towards adding new requirements and features to the product, with a fast turn-around, to fulfill customer demand and in order to position it better in the market. The system was licensed to and installed at various customer locations such as Metis, Akbank, Finansbank, Garanti, Azercell. The project was funded by TUBITAK as part of the TEYDEB programme. \\
  \multicolumn{2}{c}{} \\

  \textsc{2004-2005}            & Designed and implemented the PhonoClick Conference Server (based on PhonoClick Telephony Server) and the first iteration of the teleconferencing application that was operated both as a hosted solution by the company and as an SaaS solution in collaboration with Turkcell. Implemented various novel extensions to the VoiceXML interpreter in order to implement the conferencing system, without breaking standards compliance. \\
  \multicolumn{2}{c}{} \\


  \textsc{2005}                 & Designed and implemented the PhonoClick IP Call Center (based on PhonoClick Telephony Server) and worked on the first iteration of the ACD application package. The package provided customers with a web based console to define, administer and monitor operators, queues and calls; and, working in conjuction with PhonoClick IP Softphone, for the operators to answer (or, in some cases, initiate) calls. The system was installed at the Kamu Sertifikasyon Merkezi operated by TUBITAK in order to provide a suitable solution for their call center operation. \\
  \multicolumn{2}{c}{} \\


  \textsc{2005-2006}            & Led the Hizli Basvuru Servisi (Fast Application Service) project that was primarily developed in collaboration with Akbank to streamline their credit application process and was later deployed to other leading banks in Turkey such as Finansbank. The project involved processing credit or credit card applications through an easy-to- use phone line service by providing first-in-class features such as integrating with SOA services to lookup and utilize personal information just by providing a Citizen ID Number and accepting natural speech inputs for other pieces of complex information. The system was integrated separately with the core banking systems of each bank to query and report application statuses and to provide feedback to the customer over various channels such as outbound phone calls and text messages. As project lead, I was responsible for outlining an architecture of the various components of the system, worked with colleagues in the implementation and was point-man in technical communications between the company and the respective bank. \\
  \multicolumn{2}{c}{} \\


  \textsc{2006-2008}            & Designed and implemented the initial version of the PhonoClick Automated Outbound Calling platform. The platform worked as a cloud SaaS solution that was responsible for receiving customer data (including cell phone numbers and various customer data relevant to the call) through online channels, such as FTP/SFTP/SCP file upload or file drops, SOAP based online API; placing outbound calls to those customers; collecting customer input; and providing detailed reports of call results and customer responses to the initiating party. The principle challenges of the platform were providing high availability, efficient load distribution and fast response times which were all satisfied. The project was funded by TUBITAK as part of the TEYDEB programme. \\
  \multicolumn{2}{c}{} \\


  \textsc{Aug 2006 - Sep 2007}  & Year off from the company to complete mandatory military service. \\
  \multicolumn{2}{c}{} \\


  \textsc{2008-2010}            & Led and implemented various mobile web projects and mobile applications catering to businesses and consumers. Worked with various technologies such as LBS, HTML5, consumption and production of REST based APIs and One Time Password protocols. Implemented mobile applications for all major platforms including J2ME, Windows Mobile, BlackBerry OS, iOS and Android. Drafted and implemented a secure login process to be used in mobile applications targeted towards security-sensitive operations such as banking that provided mutual identification for both the client and the server, prevention and detection of man-in-the-middle attacks, protection of identity through multiple security layers both on the client and the server side. \\
  \multicolumn{2}{c}{} \\


  \textsc{2010-2011}            & Worked as a senior developer on the Bodru project which aims to provide a business memory platform to SMEs as a cloud solution. Architected, designed and implemented pieces in the infrastructure of the platform such as designing and setting up a scalable cloud service utilization that will enable the service to easily scale according to demand; evaluating, integrating and writing code to utilize various third-party technologies such as search engines, database abstractions, caching platforms. Evaluated, designed, implemented and refined a continous integration process for the product with a build server, automated build and test scripts, a comprehensive test suite, semi or fully automated deployment scripts and a database migration methodology and tools. \\
  \multicolumn{2}{c}{} \\

                                & Currently a partner in the company in a non-executive capacity. \\
  \multicolumn{2}{c}{} \\


  \multicolumn{2}{l}{\textbf{ParkYeri \footnotesize{Aug 2001 - Dec 2001}}} \\
  \specialrule{.01em}{0.5em}{1em}

                                & Built the first version the extensible content management system platform of the company using PHP and MySQL. The CMS was a completely customizable solution with a WYSIWYG admin interface for manipulating content and hierarchy. The system was used to provide an easy to administer web site for a local company and was deployed to production successfully. \\
  \multicolumn{2}{c}{} \\

  \newpage

  \multicolumn{2}{l}{\textbf{Turk.Net \footnotesize{Dec 2000 - Apr 2001}}} \\
  \specialrule{.01em}{0.5em}{1em}

                                & Provided consultancy as a software engineer in the Telebilet project which aimed to be the first service to provide tickets to movies via a website. The project involved working and integrating with the proprietary ticketing solutions employed by movie theathers, implementing a server backend to keep track to movie times, seat availability, seat reservation and allocation task, building a kiosk front-end to purchase and/or print tickets; and developing a web site to display current films, showings, movie theaters with ticket purchase workflow that allowed customers to pick their choice of seats among the various options offered to them. \\
  \multicolumn{2}{c}{} \\



  \multicolumn{2}{l}{\textbf{Physics Department, Bogazici University \footnotesize{Sep 1997 - Jun 2004}}} \\
  \specialrule{.01em}{0.5em}{1em}


                                & Responsible for organizing problem sessions, providing office hours, instructing lab courses, providing course lectures as stand-ins to instructors; and grading of assignments, quizes, lab work and exams. \\
  \multicolumn{2}{c}{} \\

                                & Helped setup and operated the mainframes purchased by the department to provide computational services to faculty \\
  \multicolumn{2}{c}{} \\
                                & Setup, maintained and operated the department computer lab of 10 computers with responsibilities such as formatting and installing Windows/Linux operating systems at the start of each term, troubleshooting and reporting computer problems. \\
  \multicolumn{2}{c}{} \\
                                & Organized seminars for assistants to disseminate new results from research in the department, to provide a greater understanding of the various research topics and to generate ideas for possible research areas. \\
  \multicolumn{2}{c}{} \\


\end{longtable}

%Section: Scholarships and additional info
\section{Publications}
\begin{longtable}{rl}
 \textsc{2005}  & \textit{Quantum group structures associated with invariances of some physical algebras} \\
                & U. Kayserilioglu - PhD Thesis \\
                & \textsc{Bogazici University Press} \\
                & \\

 \textsc{2003}  & \textit{Quantum invariance group of bosons and fermions} \\
                & M. Arik, U. Kayserilioglu \\
                & \textsc{hep-th/0304185} \\
                & \\

 \textsc{2002}  & \textit{The Anticommutator Spin Algebra, its Representations and Quantum Group Invariance} \\
                & M. Arik, U. Kayserilioglu \\
                & \textsc{Int.Jour.Mod.Phys.A} \\
                & \\

 \textsc{2001}  & \textit{Proceedings of the Second Gürsey Memorial Conference, Bogaziçi Univ, 19-23 June 2000} \\
                & Co-editor of the volume \\
                & \textsc{Classical and Quantum Gravity} \\
                & \\

 \textsc{2000}  & \textit{Quark and lepton masses from fundamental algebra with exact color symmetry} \\
                & E. Arik, M. Arik, U. Kayserilioglu, M. B. Unlu \\
                & \textsc{Turkish Journal of Physics} \\
                & \\

 \textsc{1999}  & \textit{Fermionic random sets and q-oscillators} \\
                & U. Kayserilioglu, J. Kornfilt, G. Unel and M. B. Unlu \\
                & \textsc{Phys.Lett.A} \\
                & \\

 \textsc{1998}  & \textit{Quantum Group description of decaying particles} \\
                & M. Arik, U. Kayserilioglu, G. Unel \\
                & \textsc{J.Phys.A: Math.Gen.} \\
                & \\
\end{longtable}

\section{Trainings and Certificates}
\begin{tabular}{rl}
 \textsc{Sep} 2006 - present & Microsoft Certified Professional               \\
 \textsc{Jun} 2010           & \textsc{SCRUM} training                       \\
 \textsc{Jun} 2009           & \textsc{W3C} Mobile Web Best Practices Course \\
\end{tabular}

%Section: Languages
\section{Languages}
\begin{tabular}{rl}
  \textsc{Turkish:} & Mothertongue \\
  \textsc{English:} & Fluent       \\
\end{tabular}

\section{Interests and Activities}
Technology, Open-Source, Programming \\
Physics, Mathematics \\
Frisbee, Bicycles, Travelling \\
\end{document}
